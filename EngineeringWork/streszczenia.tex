\newpage
\begin{center}
\large \bf
Wykorzystanie głębokich sieci neuronowych do poprawy rozdzielczości zdjęć
\end{center}

\section*{Streszczenie}
Niniejsza praca porusza problem wykorzystania ecia pecia do zrobienia czegoś wielkiego. W pracy przeanalizowane zostały algorytmy do wykrywania ecia pecia. Wybrane algorytmy zostały zaimplementowane i przebadane. Najlepsze rozwiązania zostały wykorzystane w zaprojektowanej i zbudowanej aplikacji.

\bigskip
{\noindent\bf Słowa kluczowe:} praca dyplomowa, LaTeX, jakość

\vskip 2cm

\newpage


\begin{center}
\large \bf
THESIS TITLE
\end{center}

\section*{Abstract}
This thesis presents a novel way of using a novel algorithm to solve complex
problems of filter design. In the first chapter the fundamentals of filter design
are presented. The second chapter describes an original algorithm invented by the
authors. Is is based on evolution strategy, but uses an original method of filter
description similar to artificial neural network. In the third chapter the implementation
of the algorithm in C programming language is presented. The fifth chapter contains results
of tests which prove high efficiency and enormous accuracy of the program. Finally some
posibilities of further development of the invented algoriths are proposed.

\bigskip
{\noindent\bf Keywords:} thesis, LaTeX, quality

\vfill